\chapter{شروع کار با xfce }\label{ch-3}
حالا شما مراحل نصب را پشت سر گذاشته اید و اگر پا به پای این کتاب پیش رفته باشید , در صفحه ورود فدورا قرار دارید. بعد از وارد کردن گذرواژه, وارد صفحه زیر میشوید. این همان 
\lr{\textbf{xfce}}\index{xfce}
است.
\begin{figure}[H]%ejbar shekl baray gharar gereftan zire matn
	\caption{میزکار xfce}
	\begin{center}
		\includegraphics[width=8cm]{pic/ch04/ch04-1.jpg}
	\end{center}
	\label{pic-17}
\end{figure} 
\section{کلیدهای ترکیبی کاربردی}\label{se-31}
در اینجا بصورت مختصر بعضی از کلیدهای ترکیبی و کاربردی را ذکر کنیم
\begin{table}[h]
	\caption{کلیدهای ترکیبی کاربردی}
	\begin{center}
		\begin{tabular}{|c|c|}
			\hline
			\textbf{عملیات} & \textbf{کلید ترکیبی} 
			\\
			\hline
			\hline
			باز شدن ترمینال
			\tablefootnote{در بخش های بعدی به طور کامل ترمینال تشریح خواهد شد}
			 & Ctrl+Alt+T 
			 \\
			\hline
			قفل شدن صفحه & Ctrl+Alt+Del 
			 \\
			\hline
			بازشدن پنجره ای برای جستجو برنامه های نصبی & Alt+F2 
			 \\
			\hline
			اسکرین شات گرفتن & Printscreen 
			 \\
			\hline
		\end{tabular}
	\end{center}
	\label{tab-1}
\end{table}
\section{بررسی دسکتاپ xfce}\label{se-32}
یک دستکتاپ پیش فرض 
\textbf{xfce}
شامل چند بخش است.
\begin{enumerate}
	\item منو جایی است که تمام برنامه ها در دسترس شما قرار دارند مشابه منوی استارت ویندوز
	\item نمایش دسکتاپ. با کلیلک بر روی این آیکن تمام پنجره های باز را به نوار وظیفه منتقل میکند
	\LTRfootnote{Taskbar}
	\item مرورگر وب. که در اینجا از مرورگر فایرفاکس استفاده شده است
	\item پوشه خانگی این جا همه اسناد, دانلودها, موسیقی, تصاویر و فیلم ها ذخیره میشوند
	\LTRfootnote{Home Folder}
	\item خط فرمان یا ترمینال در اینجا شما میتوانید دستورات را وارد کنید نمونه آن در ویندوز
	\lr{Command Prompt}
	است.
	\item فضای کاری. 
		\LTRfootnote{Workspaces}
	اینها مانند دسکتاپ مجازی هستند این دسکتاپ ها به  شمااجازه میدهند تا برنامه های خود را در دو یا چند دسکتاپ اجرا کنید و فضای کاری خود را افزایش دهید.
	\item تاریخ و زمان
	\LTRfootnote{system tray}
	در اینجا میتوانید تاریخ و زمان را مشاهده کنید و همچنین میتوانید تقویم ماه های فعلی را با کلیک بر روی زمان مشاهده کنید.
\end{enumerate}
\begin{figure}[H]%ejbar shekl baray gharar gereftan zire matn
	\caption{دسکتاپ xfce}
	\begin{center}
		\includegraphics[width=8cm]{pic/ch04/ch04-2.png}
	\end{center}
	\label{pic-18}
\end{figure} 
\section{بررسی منوی xfce}\label{se-33}
\begin{enumerate}
	\item در اینجا شما میتوانید نام یک برنامه را وارد کنید و منو آن برنامه را به شما در لیست نشان میدهد
	\LTRfootnote{Run Program...}
	\item خط فرمان یا ترمینال در اینجا شما میتوانید دستورات را وارد کنید نمونه آن در ویندوز
	\lr{Command Prompt}
	است.
	\item پوشه خانگی این جا همه اسناد, دانلودها, موسیقی, تصاویر و فیلم ها ذخیره میشوند
	\LTRfootnote{File Manager}
	\item ایمیل یا پست الکترونیکی احتمالاً ارزان‌ترین ابزار ارتباط از راه دور به حساب می‌آید، مخصوصاً اگر فرستنده و دریافت کننده پیام در دو کشور یا حتی دو قاره مختلف سکونت داشته باشند. امروزه برنامه‌های گوناگونی برای تسهیل فرایند ارسال و دریافت ایمیل ساخته شده اند و نرم‌افزار 
	\lr{\textbf{Mozilla Thunderbird}}\index{\lr{mozilla thunderbird}}%baray jologiri az 2 kalame shodan dar namaye
	 نیز یکی از آنها به شمار می‌رود
	 \cite{mail-reader}
	 \LTRfootnote{Mail Reader}
	 \item مرورگر وب. که در اینجا از مرورگر فایرفاکس استفاده شده است
	 \item با کلیک بر روی این نماد یک پنجره جدید باز میشود که به شما امکان خروج
	 \LTRfootnote{Log Out}
	 , راه اندازی مجدد
	 \LTRfootnote{Restart}
	 , خاموش کردن
	 \LTRfootnote{Shut Down}
	 , تعلیق
	 \LTRfootnote{Suspend}
	  کامپیوتر خود را میدهد.
\end{enumerate}
\begin{figure}[H]%ejbar shekl baray gharar gereftan zire matn
	\caption{منوی xfce}
	\begin{center}
		\includegraphics[width=4cm]{pic/ch04/ch04-3.png}
	\end{center}
	\label{pic-19}
\end{figure} 
\section{دسته بندی منو}\label{se-34}
در لینوکس فدورا, هنگامی که شما بر روی دکمه منو کلیک میکنید, برنامه ها در منو به صورت دسته بندی ذخیره میشوند. به عنوان مثال هنگامی که شما به دسته 
\textbf{Office}
بروید, همه برنامه های را مانند تصویر 
\ref{pic-20}
مشاهده میکنید.
\begin{figure}[H]%ejbar shekl baray gharar gereftan zire matn
	\caption{دسته بندی منو}
	\begin{center}
		\includegraphics[width=7cm]{pic/ch04/ch04-4.png}
	\end{center}
	\label{pic-20}
\end{figure} 
\section{تنظیمات دسکتاپ - تغییر تصویر زمینه}\label{se-35}
هنگامی که روی دسکتاپ راست کلیک میکنید منویی همانند تصویر
\ref{pic-21}
باز میشود. برای تغییر تصویر زمینه گزینه
\lr{\textbf{Desktop Settings}}\index{\lr{Desktop Settings}}\index{تغییر تصویر زمینه}
را انتخاب کنید.
\begin{figure}[H]%ejbar shekl baray gharar gereftan zire matn
	\caption{منوی راست کلیک دسکتاپ}
	\begin{center}
		\includegraphics[width=7cm]{pic/ch04/ch04-5.png}
	\end{center}
	\label{pic-21}
\end{figure}
در پنجره ای که ظاهر میشود, بر روی هر یک از تصاویر کلیک کنید به عنوان تصویر پس زمینه شما انتخاب میشود. باید متذکر شویم در فصل سفارسشی سازی به صورت کامل این موارد را بررسی میکنیم.
\begin{figure}[H]%ejbar shekl baray gharar gereftan zire matn
	\caption{تغییر پس زمینه}
	\begin{center}
		\includegraphics[width=8cm]{pic/ch04/ch04-6.png}
	\end{center}
	\label{pic-22}
\end{figure}
\section{پوشه خانگی}\label{se-36}
در لینوکس فدورا, پوشه
\textbf{Home}\index{Home}
معادل ویندوز اکسپلورر
\LTRfootnote{Windows Explorer}
است. این جایی است که تمام فایل های شما ذخیره میشوند. به عنوان مثال, هنگام بارگیری
\LTRfootnote{download}
یک فایل از اینترنت, این فایل در پوشه
\textbf{download}
بارگیری میشود. بقیه پوشه ها برای راحتی شما تنظیم شده اند که در آن میتوانید فایل های خود را بر اساس طبقه بندی آنها سازماندهی کنید. موسیقی شما به پوشه 
\textbf{Music}
, عکسهای شما به پوشه
\textbf{Pictures}
میرود و غیره. شما میتوانید فایل های خود را به طور مستقیم در پوشه خانه خود ذخیره کنید, ولی نگه داشتن فایل ها در پوشه های جداگانه باعث مرتب و دسترسی سریعتر به آنهاست. 
\begin{figure}[H]%ejbar shekl baray gharar gereftan zire matn
	\caption{پوشه خانگی}
	\begin{center}
		\includegraphics[width=10cm]{pic/ch04/ch04-7.png}
	\end{center}
	\label{pic-23}
\end{figure}
\section{مدیریت تنظیمات}\label{se-37}
در لینوکس فدورا, مدیریت تنظیمات مانند کنترل پنل ویندوز است.
\LTRfootnote{Control Panel}
ایم یک مکان مرکزی است که میتوانید تمام تنظیمات لینوکس خود را پیکربندی کنید.برای دسترسی میتوانید آدرس زیر را دنبال کنید.

\begin{tikzcd}
	Menu \ar[r, red] & Settings\ar[r, red] & Settings Manager
\end{tikzcd}
\begin{figure}[H]%ejbar shekl baray gharar gereftan zire matn
	\caption{مدیریت تنظیمات}
	\begin{center}
		\includegraphics[width=10cm]{pic/ch04/ch04-8.png}
	\end{center}
	\label{pic-24}
\end{figure}
\section{منوها}\label{se-38}
سیستم منوی در لینوکس فدورا 
\textbf{XFCE}
تا حد زیادی ساده شده است تا دسترسی و مدیریت آن بسیار سریع باشد
\subsection{منوی سیستم (system)}\label{se-381}
\begin{figure}[H]%ejbar shekl baray gharar gereftan zire matn
	\caption{منوی سیستم}
	\begin{center}
		\includegraphics[width=7cm]{pic/ch04/ch04-9.png}
	\end{center}
	\label{pic-30}
\end{figure}
\begin{description}
	\item[\lr{:Problem Reporting}] گزارش باگهای نرم افزاری 
	\item[:Gparted] مشاهده و ویرایش پارتیشن هارد دیسک \index{gparted}
	\item[\lr{:Task Manager}] یک مرور کلی از فرآینده های در حال اجرا را نمایش میدهد
	\index{task manager}
		\item[\lr{:Xfce Terminal}] خط فرمان یا ترمینال در اینجا شما میتوانید دستورات را وارد کنید.
\end{description}
\subsection{منوی آفیس (Office)}\label{se-382}
\begin{figure}[H]%ejbar shekl baray gharar gereftan zire matn
	\caption{منوی آفیس}
	\begin{center}
		\includegraphics[width=6cm]{pic/ch04/ch04-4.png}
	\end{center}
	\label{pic-31}
\end{figure} 
\begin{description}
	\item[\lr{:LibreOffice Writer}] برنامه مشابه مایکروسافت ورد
	\LTRfootnote{
		\lr{LibreOffice Writer}
\href{http://www.libreoffice.org/discover/writer/}{http://opizo.com/WCeKHT}	
}
\index{libreoffice writer}
	\item[\lr{:LibreOffice Impress}] برنامه مشابه مایکروسافت پاورپوینت
		\LTRfootnote{
		\lr{LibreOffice Impress}
		\href{http://www.libreoffice.org/discover/impress/}{http://opizo.com/kpcPD4}}
	\index{libreoffice impress}
	\item[\lr{:LibreOffice Calc}] برنامه مشابه مایکروسافت اکسل
		\LTRfootnote{
		\lr{LibreOffice Calc}
		\href{http://www.libreoffice.org/discover/calc/}{http://opizo.com/yxmbPC}}
	\index{libreoffice calc}
	\item[\lr{:Documents Viewer}] مشاهده فایل های 
	\textbf{PDF}
	\index{pdf}
\end{description}
\subsection{منوی مولتی مدیا (Multimedia)}\label{se-383}
\begin{figure}[H]%ejbar shekl baray gharar gereftan zire matn
	\caption{منوی مولتی مدیا}
	\begin{center}
		\includegraphics[width=6cm]{pic/ch04/ch04-10.png}
	\end{center}
	\label{pic-32}
\end{figure} 
\begin{description}
	\item[:‫‪brasero‬‬]\index{brasero}
	یک برنامه رایت سی دی و دی وی دی به طور کامل در بخش
	\ref{se-235}
	اقدام به نصب این برنامه کردیم
	\item[\lr{:Vlc Media Player}] 
	یک برنامه فوق العاده برای پخش فایل های صوتی و تصویری که در فصل
\ref{se-235}
	اقدام به نصب این برنامه کردیم
	بخشی از فرمتهایی که این برنامه پشتیبانی میکند
	\textbf{MKV}\index{MKV},\textbf{AVI}\index{AVI},\textbf{\lr{MP3}}\index{\lr{MP3}}
	و غیره 
			\LTRfootnote{
		\lr{Vlc Media Player}
		\href{http://www.videolan.org/vlc/index.html}{http://opizo.com/dtLXH6}}
	\index{vlc}
	\item[\lr{:PlusAudio Volume Control}] تنظیم میزان صدا به صورت کاملا مستقل برای هر برنامه در حال اجرا و همچنین میتوانید خروجی صدای خود را پیکربندی کنید.
	\index{volume control}
\end{description}
\subsection{منوی اینترنت (Internet)}\label{se-384}
\begin{figure}[H]%ejbar shekl baray gharar gereftan zire matn
	\caption{منوی اینترنت}
	\begin{center}
		\includegraphics[width=6cm]{pic/ch04/ch04-11.png}
	\end{center}
	\label{pic-33}
\end{figure}
\begin{description}
	\item[:Firefox] مرور اینترنت معادل اینترنت اکسپلورر در ویندوز
	\index{firefox}
	\item[Thunderbird] نرم افزاری برای ارسال و دریافت ایمیل این برنامه را در بخش
\ref{se-235}
نصب کردیم
\index{thunderbird}	
\end{description}
\subsection{منوی گرافیک (Graphics)}\label{se-385}
\begin{figure}[H]%ejbar shekl baray gharar gereftan zire matn
	\caption{منوی گرافیک}
	\begin{center}
		\includegraphics[width=7cm]{pic/ch04/ch04-12.png}
	\end{center}
	\label{pic-34}
\end{figure}
\begin{description}
	\item[\lr{GNU Image Manipulation program} اختصارا \lr{:Gimp}] 
	یک برنامه قدرتمند و کاملا کاربردی برای دستکاری تصویر با ویژگی هایی شبیه فتوشاپ این برنامه را در بخش
\ref{se-235}
نصب کردیم
	\LTRfootnote{
	\lr{Gimp}
	\href{http://www.gimp.org/}{http://opizo.com/S5WN65}}
\item[\lr{:Ristretto Image Viewer}] تصاویر خود را با این برنامه ببینید
\end{description}
\subsection{منوی جانبی (Accessories)}\label{se-386}
\begin{figure}[H]%ejbar shekl baray gharar gereftan zire matn
	\caption{منوی جانبی}
	\begin{center}
		\includegraphics[width=7cm]{pic/ch04/ch04-13.png}
	\end{center}
	\label{pic-35}
\end{figure}
\begin{description}
	\item[\lr{:Application Finder}] 
	منوی پاپ آپ برای پیدا کردن تمام برنامه های نصب شده تصویر
	\ref{pic-36}
	را مشاهده کنید
	\item[\lr{:Xarchive}] 
	برنامه ای برای فشرده سازی و بالعکس فایل های زیپ و غیره
	\item[:Calculator] ماشین حساب
	\item[\lr{:catfish File Search}] جستجو بین فایل ها و پوشه ها
	\item[\lr{:Screenshot}] عکس گرفتن از دسکتاپ 
	\item[:Terminal] ترمینال مکانی است برای وارد کردن دستورات مختلف سیستم
	\item[:Leafpad] یک ویرایشگر متنی ساده است همانند 
	\textbf{notepad}
	ویندوز	
\end{description}
\begin{figure}[H]%ejbar shekl baray gharar gereftan zire matn
	\caption{یافتن برنامه}
	\begin{center}
		\includegraphics[width=7cm]{pic/ch04/ch04-14.png}
	\end{center}
	\label{pic-36}
\end{figure}